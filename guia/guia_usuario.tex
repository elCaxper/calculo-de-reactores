\documentclass[20pt,a4paper]{extarticle}
\usepackage[a4paper,margin=6mm]{geometry}
\usepackage[spanish]{babel}
\usepackage{amsmath}
\usepackage{hyperref}
\usepackage[utf8]{inputenc}

\title{Diseño de reactores ideales}
\author{Gustavo Plaza Roma y Jesús Casado Gonzalez}
%\geometry{paperheight=5000pt} % Lo vamos a usar para que solo haya una página,
\begin{document}

\maketitle

\setcounter{tocdepth}{2} % Para no mostrar subsubsections en el indice
%\tableofcontents

\section{Introducción}
	Esta aplicación se ha desarrollado usando Python 3, la interfaz gráfica se ha creado con el programa \textit{Qt Designer} y se ha transformado a Python para que sea compatible con la librería \textit{PySide 1.2.4}. 
	
	Para el correcto funcionamiento de la aplicación es necesario tener los siguientes módulos instalados:
	\begin{itemize}
		\item Pyqtgraph para poder visualizar gráficas dentro de la aplicación Qt.
		\item Matplotlib para poder visualizar y exportar los resultado de los cálculos.
		\item Scipy permite realizar las distintas operaciones matemáticas.
		\item Numpy proporciona las clases necesarias para operar con vectores.
	\end{itemize}


\section{Reactor discontinuo}
	
	Son aquellos que trabajan por cargas, es decir se introduce una alimentación, y se espera un tiempo dado, que viene determinado por la cinética de la reacción, tras el cual se saca el producto.
	
	\subsection{Reactor discontinuo isotermo}
		La ecuación que proporciona el tiempo de reacción en este modo de operación es:
		
		\begin{equation*}
			t = C_{A0}\int_{X_{A0}}^{X_A}\frac{dX_A}{(-r_a)}
		\end{equation*}
		
		Donde $(-r_a)$ se puede calcular como:
		
		\begin{equation*}
			(-r_a) = K \cdot C_{A0}^n (1-X_A)^n; ~~~~~ K = K_0\exp\left(\frac{-E_a}{RT}\right)
		\end{equation*}
		
		Según el orden de reacción el tiempo se podrá calcular como:
		
		\begin{itemize}
			\item Orden 0: $ t = \frac{C_{A0}}{K}(X_A-X_{A0})$
			\item Orden 1: $ t = \frac{1}{K}\ln\left(\frac{1-X_{A0}}{1-X_A}\right)$
			\item Orden 2: $ t = \frac{1}{KC_{A0}}\left(\frac{X_A}{1- X_A}\right)$
		\end{itemize}
	
	
	\subsection{Reactor discontinuo adiabático}
		La ecuación que proporciona el tiempo de reacción en este modo de operación es:
	
		\subsubsection{Balance de materia}
			\begin{equation*}
			t = C_{A0}\int_{X_{A0}}^{X_A}\frac{dX_A}{(-r_a)}
			\end{equation*}
			
			Donde $(-r_a)$ se puede calcular como:
			
			\begin{equation*}
			(-r_a) = K \cdot C_{A0}^n (1-X_A)^n; ~~~~~ K = K_0\exp\left(\frac{-E_a}{RT}\right)
			\end{equation*}
			
		\subsubsection{Balance de energía}
			\begin{equation*}
				T = T_0 + \frac{(-\Delta H_R)C_{A0}}{\rho C_p}(X_A-X_{A0})
			\end{equation*}
			
	\subsection{Reactor discontinuo no discontinuo y no adiabático}
		La ecuación que proporciona el tiempo de reacción en este modo de operación es:
		
		\subsubsection{Balance de materia}
			\begin{equation*}
				\frac{dt}{dX_A} = \frac{C_{A0}}{K_0 \exp\left(\frac{-E}{RT}\right) C_{A0}^n (1-X_A)^n}
			\end{equation*}
			
		\subsubsection{Balance de energía}
			\begin{equation*}
				\frac{dT}{dX_A} = \frac{(-\Delta H_R)C_{A0}}{\rho C_p} + \frac{C_{A0}US(T_c-T)}{V\rho C_p K_0 \exp\left(\frac{-E_a}{RT}\right)C_{A0}^n(1-X_A)^n}
			\end{equation*}
			
\section{Condiciones óptimas}
	\subsection{Conversión óptima}
		\begin{equation*}
			X_{A_{opt}} = 1- \frac{C_R a}{(\Delta w) C_{A0}VK}
		\end{equation*}		
		
	\subsection{Tiempo óptimo de reacción}
		\begin{equation*}
			t_{opt} = \frac{1}{K}\ln \left(\frac{(\Delta w)C_{A0}VK}{C_R a}\right)
		\end{equation*}
		
\section{Reactor continuo}
	\subsection{Cálculo del volumen}
		\subsubsection{Balance de materia}
			\begin{equation*}
				V= \frac{X_{AF}v_0C_{A0}}{(-r_a)_f}
			\end{equation*}
			
			Donde $(-r_a)_f$ se calcula como:
			
			\begin{equation*}
				(-r_a)_f = K_fC_{A0}^n (1-X_{Af})^n; ~~~~~ K_f = K_0 \exp \left(\frac{-E_a}{RT_f}\right)
			\end{equation*}
	
	\subsection{Cálculo de la conversión}
		\subsubsection{Balance de materia}
			\begin{itemize}
				\item Orden 0: $X_{Af} = \frac{VK_f}{v_0C_{A0}}$
				\item Orden 1: $X_{Af} = \frac{VK_f}{v_0+K_fV}$
				\item Orden 2: $X_{Af} = \frac{\left(\frac{v_0}{VK_fC_{A0}}+2\right)\pm \sqrt{\left(\frac{v_0}{VK_fC_{A0}}+2\right)^2-4}}{2}$
			\end{itemize}
\end{document}
